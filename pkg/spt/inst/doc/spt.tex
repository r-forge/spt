
\documentclass{article}
\usepackage{graphicx}
\usepackage[colorlinks=true,urlcolor=blue]{hyperref}

\usepackage{color}

\usepackage{Sweave}
\newcommand{\strong}[1]{{\normalfont\fontseries{b}\selectfont #1}}
\let\pkg=\strong

\title{ S4 Classes and Methods for Spatiotemporal Data:\\
the {\tt spt} Package }
\author{J. Blair Christian\footnote{US EPA,
Research Triangle Park, NC {\tt blair.christian@gmail.com}}}
\date{15 October 2009}

\begin{document}
\maketitle
\tableofcontents

\section*{Introduction}

Note: This vignette was directly copied from the sp package, then
edited.  Hopefully they will see this as flattery...

The {\tt spt} package provides S4 classes and methods for dealing with
spatiotemporal data in S (R and S-Plus\footnote{our primary efforts target R;
depending on the needs, we will address S-Plus as well}). The spatiotemporal
data structures implemented currently include points, with support for
lines, polygons and grids coming in the next release, 0.01-02.  We have chosen to use S4
classes and methods style (Chambers, 1998) to allow validation of objects
created. Although we mainly aim at using spatiotemporal data in the geographical
(two-dimensional) domain, the data structures that have a straightforward
implementation in higher dimensions (points, grids) do allow this.

The motivation to write this package was born from my work at the EPA as
a postdoc along with \href{the SAMSI program on Spatiotemporal
  analysis}{http://www.samsi.info/programs/2009spatialprogram.shtml}
where I had the privilege to be a party crasher.  At this time, there
are no packages with broad support for spatiotemporal data, however
there are various associated 

The package is available on R-forge. From the
package home page, \url{http://spt.r-forge.r-project.org}, a graph
gallery with R code, and the development source tree are available.

This vignette describes the initial classes, methods and functions provided
by \texttt{spt}. Instead of manipulating the class slots (components) directly,
we provide methods and functions to create the classes from elementary
types such as data.frames or vectors and to convert them back
to any of these types. Also, coercion (type casting) from one class to
the other is provided sparsely, where relevant.  If something isn't
provide, please request it from the author(s).

Package {\tt spt} is installed by 
\begin{Schunk}
\begin{Sinput}
> install.packages("spt", repos = "http://R-forge.R-project.org")
\end{Sinput}
\end{Schunk}

Currently, I put a lot of dependencies and they're not all necessary
now (especially the suggested packages).  

Package {\tt spt} is loaded by 
\begin{Schunk}
\begin{Sinput}
> library(spt)
\end{Sinput}
\begin{Soutput}
 Try help(fields) for an overview of this library 
 [1] "spt"       "fda"       "zoo"       "splines"   "maps"      "nlme"     
 [7] "RCurl"     "bitops"    "timeDate"  "fields"    "spam"      "sp"       
[13] "tools"     "stats"     "graphics"  "grDevices" "utils"     "datasets" 
[19] "methods"   "base"     
\end{Soutput}
\end{Schunk}


\section*{Spatiotemporal Data Classes}
Spatiotemporal data is a complicated affair.  There are many, many
types of spatiotemporal data and many, many uses of it. I do not
presuppose to help all users with all applications.  The current
efforts only pertain to my current work with air pollution data,
monitoring sites located at lat/long locations $(x,y)$, producing an
observation such as the measurement of ozone in ppm at regular time
intervals (eg hours, days, etc), usually which have some relation to
the regulatory standard (I have the outline of functions to calculate the
primary and secondary ozone standard, for example).  These stations eventually produce
flat (eg text) files which are 4 columns (lat, long, time, obs) and
$n_{obs}$ rows.


\subsection*{Spatiotemporal Classes}
The basic structure of spatiotemporal classes consists of 3 core
classes and 1 optional class.  The core classes are for storing the
temporal part of the data, the spatial part of the data, and the
observations.  These are stored somewhat independently, with a unique
identifier for the time and spatial location providing the map between
the classes, in the spirit of a relational database.  This has several
advantages. First, the bulky temporal and spatial parts are reduced
to their unique elements, reducing storage space and speeding up
computation by performing operations on the unique elements, then
duplicating those results as needed.  Second, it allows the data to be
stored in different places.  This is not currently implemented, but it
would allow the spatial parts to be stored using, say, the
\texttt{SQLiteMap} package for the spatial part.  

The current class is \texttt{SpatialTemporalDataFrame}.  However, only
point data is currently (v 0.01-01) supported.  

Currently, this class contains objects of the other classes.  From a
big picture view, I don't know if it's better to use
\texttt{setClassUnion} instead.

\subsection*{Temporal Classes}
Currently, the temporal class is called \texttt{stTemporal}.  It
consists of a pairlist (but it's currently 2 vectors, the integer
unique temporal IDs, and the timeDate time stamps.

The format is very important to ensure that the timedates are read
correctly.  Please visit the help page for \texttt{strptime} for the
complete list.

Example: create, get,
\begin{Schunk}
\begin{Sinput}
> tmp <- stTemporal(timeVals = c("2008-09-29", "2009-01-14", "2012-12-12"), 
+     format = "%Y-%m-%d")
\end{Sinput}
\begin{Soutput}
A vector of timeDates of length 3 .
\end{Soutput}
\begin{Sinput}
> tmp
\end{Sinput}
\begin{Soutput}
A vector of timeDates of length 3 .
\end{Soutput}
\begin{Sinput}
> getTid(tmp)
\end{Sinput}
\begin{Soutput}
[1] 1 2 3
\end{Soutput}
\begin{Sinput}
> getTimedatestamps(tmp)
\end{Sinput}
\begin{Soutput}
[1] "2008-09-29" "2009-01-14" "2012-12-12"
\end{Soutput}
\end{Schunk}


\subsection*{Spatial Classes}
Currently, the spatial class is called \texttt{stSpatial}.  It
consists of 2 objects, the integer
unique spatial IDs, and a \texttt{Spatial} object, currently only
\texttt{SpatialPoints} is supported.

Please visit the vignett for the \texttt{sp} package for the
complete list of items to be supported in version 1.  Note that none
of the data containing classes will be supported here since we have
our own data classes.

Currently I inherit the spatial class, and I'm not sure what the
implications are for that (there's a big difference whether you
contain an object of a class or whether you extend a class).

Example: create, get,
\begin{Schunk}
\begin{Sinput}
> n <- 10
> d <- 2
> crd <- matrix(runif(n * d), n, d)
> bbox <- cbind(apply(crd, 2, min), apply(crd, 2, max))
> colnames(bbox) <- c("min", "max")
> tmp <- new("stSpatialPoints", s.id = as.integer(1:n), coords = crd, 
+     bbox = bbox, proj4string = CRS(as.character(NA)))
> tmp
\end{Sinput}
\begin{Soutput}
A collection of 10 points.
\end{Soutput}
\begin{Sinput}
> getSid(tmp)
\end{Sinput}
\begin{Soutput}
 [1]  1  2  3  4  5  6  7  8  9 10
\end{Soutput}
\begin{Sinput}
> getSpatialPoints(tmp)
\end{Sinput}
\begin{Soutput}
SpatialPoints:
           [,1]      [,2]
 [1,] 0.6739453 0.3021973
 [2,] 0.9619853 0.6170330
 [3,] 0.9228386 0.9050982
 [4,] 0.7747456 0.8476483
 [5,] 0.7248062 0.4579902
 [6,] 0.7413480 0.3863993
 [7,] 0.6302933 0.6443511
 [8,] 0.1903071 0.7170724
 [9,] 0.6977443 0.1980982
[10,] 0.3703263 0.2772779
Coordinate Reference System (CRS) arguments: NA 
\end{Soutput}
\end{Schunk}


\subsection*{Data Classes}
The main data class is called \texttt{stDataFrame}
\begin{Schunk}
\begin{Sinput}
> tmp <- new("stDataFrame", s.id = as.integer(1:3), t.id = as.integer(1:3), 
+     df = data.frame(s.id = as.integer(1:3), t.id = as.integer(1:3), 
+         x1 = runif(3), x2 = rnorm(3)))
> tmp
\end{Sinput}
\begin{Soutput}
A dataframe with 3 observations and 2 data columns with
 3 unique timedates and
 3 unique locations.
\end{Soutput}
\end{Schunk}

\subsection*{Metadata Classes}
Originally I had put in a metadata class to provide information about
the data: units, collection data, url, citation info, etc, but I have
removed it until the package is more mature unless there is rampant
support. 


\section*{Methods}
There are methods for: dist, join, subset, apply, but I don't have
time to fully go into them yet (short on time)....

\section*{Quick Tour}

\begin{Schunk}
\begin{Sinput}
> pollutant <- "Oz"
> state <- c("SC", "NC")
> startDate <- "2007-01-01"
> endDate <- "2007-01-30"
> tmp <- getEPAAirExplorerData(pollutant, state, startDate, endDate)
\end{Sinput}
\begin{Soutput}
[1] "getting data for state SC ..."
[1] "getting data for state SC ..."
[1] "getting data for state NC ..."
\end{Soutput}
\begin{Sinput}
> tmp
\end{Sinput}
\begin{Soutput}
A dataframe with 554 observations and 3 data columns with
 30 unique timedates and
 11 unique locations.
\end{Soutput}
\begin{Sinput}
> plot(tmp)