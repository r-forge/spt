
\documentclass[12pt]{article}
\title{Thoughts on Spatiotemporal \\
  Data Standards in R}
\author{Blair Christian\\
\texttt{blair.christian@gmail.com}}
\date{}

\usepackage{natbib,hyperref,fullpage,amsmath}

%%%%%%%%%%%%%%%%%%%%%%%%%%%%%% User specified LaTeX commands.

\begin{document}
\maketitle

\section*{Introduction}
Spatiotemporal data is any set of observations which have space and
time components.  There are rich tool kits in R for working on either
spatial or temporal data, but few which support spatiotemporal data.
Currently, it is necessary to manually slice and dice your
spatiotemporal data in order to apply these existing R tools.  This
lack of infrastructure in the S4 object sense has greatly hindered the
use and development of spatiotemporal modeling.  This is a proposal
to add a small amount of structure in the form of S4 objects in order
to leverage the existing spatial and temporal packages in R in a
seamless manner, in addition to creating an environment conducive to
code reuse.  I am thinking about this with a possible future endpoint in
the spirit of \href{http://www.bioconductor.org}{bioconductor} or \href{http://www.rmetrics.org/}{Rmetrics}.  I'll
put a copy of this up on an informal website,
\href{http://www.isds.duke.edu/~jbc30/SpatioTemporal/}{http://www.isds.duke.edu/~jbc30/SpatioTemporal/}
and move it to \href{http://r-forge.r-project.org}{R-forge} as we develop
this.  


\section*{Audience}

\subsection*{End User}
As an end user, some use cases include loading the data one time, then
calling one or more models from one or more packages.  For example,
being able to run both, say, spBayes models and INLA models, in order
to compare various computational and model properties.  (eg
performance in practice, comparison of predictive errors or other
quantities of interest)


\subsection*{Model Maker}
As a model maker, I don't want to reinvent the wheel.  If other
packages provide adequate functions for some needed algorithm, a data
standard should help especially in pre and possibly in post
processing. 

\subsection*{Data Provider}
As a data provider, one would like to provide data in an easy to
obtain, easy to work with and be easy to cite.  With Amy Braverman's
comments in mind from
\href{http://www.samsi.info/workshops/2009spatial-opening200909.shtml}{the
  recent SAMSI conference}, if NASA could provide data access similar to the
\href{http://www.ncbi.nlm.nih.gov/Entrez/}{NIH's entrez system}.
(something in process now, they have
\href{http://www-misr.jpl.nasa.gov/mission/data/getdata.html}{almost 4 Pb
  of imaging data at NASA}, most of which isn't currently available.
Metadata allowing the data provider room for descriptions of how the
data were collected, who to contact with questions, a website for more
information and a citation for using the data also 


To summarize the benefits of a spatiotemporal classes and methods:

\begin{itemize}
\item Reproducible data analysis (see the
  \href{http://www.bioconductor.org/docs/faq/index.html#Documentation\%20and\%20reproducible\%20research}{bioconductor
  policy} of R's \href{http://www.bioconductor.org}{bioconductor project})
\item increase code reuse
\item increase productivity
\item possibility for better context/information for datasets (a metadata requirement)
\end{itemize}



\section*{Spatiotemporal Data Standards}

Some general needs for a data representation include:

\begin{itemize}
\item data import from flat files or relational database
\item structure for metadata (see bioconductor)
\begin{itemize}
\item creator, maintainer, location on web, contact info,
algorithm/satellite used, dates, original purpose/funding,
citation of data 
\end{itemize}
\item change of support tools, foe example applied to upscaling/downscaling
\begin{itemize}
\item (methods for sp, which could then be extended to spatiotemporal)
\end{itemize}
\item registration?
\item data export and visualization
\begin{itemize}
\item export in kml format for portability
\item google maps, for example see
  \href{http://googlemapsmania.blogspot.com/2009/04/swine-flu-outbreak-on-google-maps.html}{swine
  flu tracker}
\item other types of visualization relevant for data, eg trip library
  for following the trips of tagged animals
\item
  \href{http://www.isds.duke.edu/~jbc30/Presentations/movie.gif}{generic
  movie showing how grid data- kriging results in this example- change
  over time}, via the animation package \citep{xie09}
\end{itemize}

\item Methods for accessing data
\begin{itemize}
\item time series by location
\item spatial data at a time
\item function of all spatial or all temporal (eg mean, max, etc) at a
  given location/time (again, something that could be implemented for
  Spatial class and inherited by a spatiotemporal class)
\end{itemize}


\end{itemize}

There are a few assumptions I have been making, including:
\begin{itemize}
\item R is the right place
\item can handle big datasets (with out of core/db storage)
\item easier/more convenient to call ``real'' code in C/fortran, etc
  from R than rolling ones own
\end{itemize}


There are some risks
\begin{itemize}
\item too much structure
\item too hard to use
\item audience not ready to adopt (not seen as a problem yet)
\end{itemize}

\subsection*{Types of Spatiotemporal Data}
The combinatorics of spatiotemporal data are large, and may increase with
change of support methods.  I really don't have any idea what the best
way to proceed here is, but I have a few stabs that aren't so
elegant.  Thoughts?  What do you want supported asap on the spatial
side?  temporal side?

I have one set of thoughts about the high level design (see section below)
\begin{itemize}
\item store data in 3 parts, link with unique identifiers; spatial, temporal, data observations
\item spatial extends/implements Spatial in sp package
  \citep{pebesma05}
\item temporal part extends/implements a time series or fda or splines
  or $\ldots$ package
\end{itemize}

There is a bit of existing code that should serve as some
representation of the need for spatiotemporal data in different areas:
\begin{itemize}
\item spBayes \citep{finley09},
  \href{http://www.math.ntnu.no/~hrue/inla/}{inla} (example of models
  sharing input format that could be compared)
\item geoR \citep{ribeiro01}, fields \citep{furrer09}, etc (retrieve those spatial objects at a given time)
\item tseries, fda \citep{ramsay09}, its \citep{its09}, zoo
  \citep{zeileis05}, xts \citep{ryan09} etc (retrieve those temporal objects at a given location)
\item leverage code (eg bioBase \citep{gentleman04})- if there's a package for spatial epi
that is useful (eg trip \citep{sumner09}), then we should embrace code reuse instead of
reinventing the wheel 
\end{itemize}

\subsection*{Wishlists}
Here are some thoughts that are random and not comprehensive.  Email
me to add some.

\subsubsection*{Short Term Wishlist}
\begin{itemize}
\item focus on data storage
\begin{itemize}
\item easy to store
\item easy to retrieve
\item some basic EDA for spatiotemporal
\begin{itemize}
\item movies
\item standard epi plot over time? (epi spatial package?)
\item seamless spatial views at time
\item seamless temporal views at location
\end{itemize}
\end{itemize}
\item S4 classes and methods
\item aid dissemination (google maps display)
\item export to standard format (kmz? for google maps, GRASS, arcGIS, $\ldots$)
\item examples?
\item vignette?
\item start small? (regular point/grid?)
\end{itemize}


\subsubsection*{Long Term Wishlist}
\begin{itemize}
\item database support for massive (out of core) datasets
\item advice to data providers (eg NASA, EPA, etc) for online data access (see NIH stuff, entrez)
\item people who's advice I would like from an OO + R perspective: Bioconductor team (Seth Falcon?
bioBase authors?); Dirk Eddelbuettel; Dough Bates; whoever reads this far $\ldots$
\end{itemize}



\section*{Example Class}

An example starting place

I'm working with monitoring data at the EPA right now for 1-3
gasses.  Some sites only monitor 1 gas at one set of intervals (say,
hourly).  Another, overlapping set of sites measures other gasses at
other intervals (say, every three days).  From this data, I use a
variety of kriging and related methods to interpolate and predict gas
levels at locations in between the sites, and at future points in
time.  The approaches I'm using are pretty much out of the
\citet{banerjee2004hma}, so a
big assumption is separability in temporal and spatial covariances.
In my case, space is treated as continuous and I'll be treating time
as regular for now.  My classes will be focused only on this in the
short term. I'll put up some info on the classes I use next week or
the week after for those of you who are interested. 



Here are some first thoughts at a very general (eg "sp" library
    type) approach to classes for spatiotemporal data.  Please let me
    know if I have made any gross mistakes here- it's been a long
    while since I played with S4 classes.  My first thoughts are to
    have a main class, SpatialTemporalDataFrame, which has a unique
    set of spatial objects, a unique set of temporal objects, and a
    dataframe with the data that maps to a unique spatial location and
    time.  My first go is something like this:




\begin{verbatim}

## unique set of spatial locations
##
## basically a container for one or more points or polygons
##
## eg something from the sp package,
##    such as SpatialPoints, SpatialPixels, SpatialPolygons
library(sp)
setClass("stSpace",
         representation(s.id="integer",
                        location="Spatial"))

validstSpaceObject <- function(object) {
  if(length(object@s.id) == length(object@location)) TRUE
  else paste("Unequal s.id, location lengths: ", length(object@s.id), ", ",
             length(object@location), sep="")
}
## assign the function as the validity method for the class
setValidity("stSpace", validstSpaceObject)


## I guess you could make it something like a
## SpatialPointsDataFrame, and put the unique ID in the data frame?
## but for generality, it would be nice to have a matching structure below
## for the time part


## unique set of time stamps
## basically a container for a time and date stamp.
library(timeDate)
setClass("stTime",
         representation(t.id="integer",
                        timedate="timeDate"))

validstTimeObject <- function(object) {
  if(length(object@t.id) == length(object@timedate)) TRUE
  else paste("Unequal t.id, timedate lengths: ", length(object@t.id), ", ",
             length(object@timedate), sep="")
}
## assign the function as the validity method for the class
setValidity("stTime", validstTimeObject)


## a dataframe subsettable by (unique) spatial and/or temporal ids
setClass("stDataFrame",
         representation(s.id="integer",
                        t.id="integer",
                        df="data.frame"))


## Seems like these should all be virtual classes?
##
## basically a mini relational database
## possibly stored that way in the future for large datasets
## see: RSQLite, RSQLiteMap packages
setClass("SpatialTemporalDataFrame",
         representation(spatial="stSpace",
                        temporal="stTime",
                        data="stDataFrame",
			metadata="list"))

\end{verbatim}

%\begin{figure}
%\begin{center}\includegraphics[  width=4in,
%  height=3in]{network}\end{center}

%\begin{center}Figure 1: Campus Networks (Unix)\end{center}
%\end{figure}


%\begin{figure}[h]
%  \begin{center}
%    \includegraphics[scale=0.3,angle=0]{mid.eps}
%    \includegraphics[scale=0.3,angle=0]{nwCond.eps}\\
%    \includegraphics[scale=0.3,angle=0]{nwWeights.eps}
%    \includegraphics[scale=0.3,angle=0]{nwCondWeight.eps}

%    \caption{Regular, Conditioning, Weighting, Conditioning and Weighting}
%  \end{center}
%\end{figure}

\section*{Conclusion}
Thoughts?

\bibliography{spatial}
\bibliographystyle{plainnat}


\end{document}
